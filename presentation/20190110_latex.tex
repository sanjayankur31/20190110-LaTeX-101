%% Ankur Sinha
\documentclass[usenames,dvipsnames]{beamer}

%% packages %%
% support for coloured text
\usepackage{color}
\usepackage[scale=2]{ccicons}
\usepackage{amssymb}
\usepackage{jneurosci}
\usepackage{subfig}
\usepackage[T1]{fontenc}
\usepackage[utf8]{inputenc}
% Use opensans
\usepackage[default,osfigures,scale=0.95]{opensans}
% for strike through
\usepackage[normalem]{ulem}
% links, urls, refs
\definecolor{links}{HTML}{2A1B81}
% Fedora blue for the theme
\definecolor{FedoraBlue}{HTML}{2A1B81}
\usepackage{hyperref}
\hypersetup{colorlinks,linkcolor=Green,urlcolor=links}
% graphics
\usepackage{graphicx}
% algorithm
\usepackage{textcomp}
\usepackage{wrapfig}
\usepackage{textgreek}
\usepackage{euler}
\usepackage{minted}

% beamer theme
% use defaults for theme
\usetheme[numbering=fraction]{metropolis}
\usefonttheme[onlymath]{serif}
\setbeamerfont{footnote}{size=\tiny}
\setbeamerfont{caption}{size=\tiny}
\setbeamercolor{alerted text}{fg=Green}

% Not needed in metropolis, but in general footnote citation fixes: https://tex.stackexchange.com/questions/44217/how-can-i-stop-footcite-from-hijacking-my-beamer-columns
% how to use multiple references to the same footnote: https://tex.stackexchange.com/questions/27763/beamer-multiple-references-to-the-same-footnote

%% title %%
\title[\LaTeX{}]{\LaTeX{} 101}
\subtitle{\LaTeX{} for absolute beginners}
\author[ankursinha: FranciscoD]{Ankur Sinha @ Fedora}
\date{January 10, 2019}

%% document begins %%
\begin{document}

% title frame %%
\begin{frame}
  \titlepage{}
\end{frame}

%% Three slides for 5 minutes seems good
\section{Setup}
\begin{frame}[c]{Before we begin}
  \begin{itemize}
    \item Please turn off your web cam, and mute your microphone.
    \item Please install the required packages:\\ \texttt{sudo dnf install texlive-latex texlive-bibtex git}
  \end{itemize}
\end{frame}
\section{\LaTeX{}?}
\begin{frame}[c]{What is \LaTeX{}?}
  \TeX{} (pronounced \emph{tech})\footnotemark:
  \begin{itemize}
    \item a \alert{typesetting} program:
      \begin{itemize}
        \item typesetting: formatting documents (articles, presentations, posters, pamphlets \ldots).
        \item focusses on appearance, rather than structure/content.
      \end{itemize}
  \end{itemize}
  \pause{}
  \LaTeX{} (pronounced \emph{la-tech})\footnotemark:
  \begin{itemize}
    \item set of \alert{macros} (commands) built over \TeX{}:
      \begin{itemize}
        \item makes it easier to use, to quickly carry out common tasks.
        \item lets the author focus on content.
      \end{itemize}
  \end{itemize}
  \footnotetext[1]{\href{https://en.wikipedia.org/wiki/TeX\#Pronunciation_and_spelling}{Pronunciation of \TeX{}.}}
  \footnotetext[2]{\href{http://www.tug.org/levels.html}{Levels of \TeX{}.}}
\end{frame}

\begin{frame}[c]{Why should one use \LaTeX{}?}
  \LaTeX{}:
  \begin{itemize}
    \item is \href{https://u.fsf.org/user-liberation}{Free software}: it is free to use, share, modify, and study:
      \begin{itemize}
        \item enables \href{https://en.wikipedia.org/wiki/Open_science}{Open Science}.
      \end{itemize}
    \item is available on Linux, MacOSX, and Windows.
    \item is easy to use (as you will see).
      \pause{}
    \item has minimum requirements:
      \begin{itemize}
        \item is written in \alert{plaintext}: so \alert{any} editor will do, and \alert{version control} can be used (git!),
        \item required \LaTeX{} packages.
      \end{itemize}
      \pause{}
    \item generates easy to read, clean documents\footnotemark.
  \end{itemize}
  \footnotetext[3]{This presentation is written in \LaTeX{}}
\end{frame}
\section{Using \LaTeX{}}
\begin{frame}[c]{Requirements}
  \begin{itemize}
    \item a text editor, \alert{any} text editor\footnotemark: vim/emacs/gedit/atom/geany/\ldots{}.
    \item \LaTeX{} packages.
      \pause{}
    \item IDEs speed up writing:
      \begin{itemize}
        \item \href{https://www.overleaf.com}{Overleaf} (web based).
        \item \href{http://www.xm1math.net/texmaker/}{Texmaker}.
        \item \href{https://www.lyx.org/}{LyX}.
        \item Vim/Emacs/Atom/\ldots{} + plug-ins.
      \end{itemize}
  \end{itemize}
  \footnotetext[4]{Microsoft Word is not an editor. It is a text processor---it does not generate plaintext as output.}
\end{frame}
\begin{frame}[fragile]{Document structure}
  \begin{figure}
    \centering
    \begin{minted}[fontsize=\footnotesize]{latex}
% Comments start with a %
% Comment blocks are not supported!
\documentclass{...}

% This region is called the preamble.
% Commands that setup the whole document go here.
% Such as inclusion of packages.

\begin{document}

% The text of the document goes here.

\end{document}
% Anything here will not be processed.
    \end{minted}
  \end{figure}
\end{frame}
\begin{frame}[fragile]{Document classes}
  \begin{itemize}
    \item standard layout for \LaTeX{} to use for the whole document.
    \item classes are provided as \texttt{.cls} files.
      \pause{}
    \item standard classes:
      \begin{itemize}
        \item article.
        \item report.
        \item letter.
        \item \ldots{}
      \end{itemize}
    \item packages provide more:
      \begin{itemize}
        \item beamer (used to make this presentation).
      \end{itemize}
  \end{itemize}
  \pause{}
  \begin{figure}[h]
    \centering
    \begin{minted}[fontsize=\footnotesize]{latex}
% Let us write an article
\documentclass[a4paper]{article}
    \end{minted}
  \end{figure}
\end{frame}
\begin{frame}[fragile]{The preamble}
  \begin{itemize}
    \item Commands for the whole document: commonly: inclusion of packages.
      \begin{itemize}
        \item Packages provide extra functionality\footnotemark.
      \end{itemize}
  \end{itemize}
  \begin{figure}[h]
    \centering
    \begin{minted}[fontsize=\footnotesize]{latex}
% Use Opensans fonts
\usepackage[default,osfigures,scale=0.95]{opensans}
% Use the hyperref packages for better links
\usepackage{hyperref}
% A title?
\title{An example \LaTeX{} document}
% The author?
\author{A. Human}
    \end{minted}
  \end{figure}
  \footnotetext[5]{\href{https://ctan.org/}{Search CTAN for packages, and documentation}}
\end{frame}
\begin{frame}[fragile]{The text!}
  \begin{figure}[h]
    \centering
    \begin{minted}[fontsize=\footnotesize]{latex}
% Let LaTeX format the title
\maketitle
% Some text?
\LaTeX{} is easy!
    \end{minted}
  \end{figure}
\end{frame}
\begin{frame}[fragile]{The source file so far}
  \begin{figure}[h]
    \centering
    \begin{minted}[fontsize=\footnotesize]{latex}
% Filename: example-doc.tex
\documentclass[a4paper]{article}
\usepackage[default,osfigures,scale=0.95]{opensans}
\usepackage{hyperref}

\title{An example \LaTeX{} document}
\author{A. Human}

\begin{document}
\maketitle

\LaTeX{} is easy!

\end{document}
    \end{minted}
  \end{figure}
\end{frame}
\begin{frame}[fragile]{Generating the document: I}
  \begin{figure}[h]
    \centering
    \begin{minted}[fontsize=\footnotesize]{console}
    $ ls
    example-doc.tex
    $ latex example-doc
    ...
    $ ls
    example-doc.aux  example-doc.dvi  example-doc.log
    example-doc.out  example-doc.tex
    \end{minted}
  \end{figure}
  \begin{itemize}
    \item No PDF file?
  \end{itemize}
\end{frame}
\begin{frame}[fragile]{Generating the document: II}
  \begin{itemize}
    \item The default output from \LaTeX{} is DVI (Device independent file format).
    \item But, PDF is most common now. So:
  \end{itemize}
  \begin{figure}[h]
    \centering
    \begin{minted}[fontsize=\footnotesize]{console}
    $ pdflatex example-doc
    ...
    $ ls
    example-doc.aux  example-doc.dvi  example-doc.log
    example-doc.out  example-doc.pdf  example-doc.tex
    \end{minted}
  \end{figure}
  \begin{itemize}
    \item View it with your favourite PDF viewer (Evince/Okular/Adobe/Zathura/\ldots{})
  \end{itemize}
\end{frame}
\begin{frame}[fragile]{Text structures: sections, lists}
  \begin{figure}[h]
    \centering
    \begin{minted}[fontsize=\footnotesize]{latex}
\section{Our first section}
This is a new section.

\subsection{Lists}
\begin{itemize}
  \item An itemised list!
\end{itemize}

\begin{enumerate}
  \item A numbered list
\end{enumerate}
    \end{minted}
  \end{figure}
  \begin{itemize}
    \item Save, re-run \texttt{pdflatex}, view.
  \end{itemize}
\end{frame}
\begin{frame}[fragile]{Text structures: mathematics}
  \begin{figure}[h]
    \centering
    \begin{minted}[fontsize=\footnotesize]{latex}
\begin{equation}
  h^2 = b^2 + p^2
\end{equation}
where \(h\), \(b\), and \(p\) are the lengths of the
hypotenuse, the base, and the perpendicular of a
right angled triangle respectively.
    \end{minted}
  \end{figure}
  \begin{itemize}
    \item Save, re-run \texttt{pdflatex}, view.
  \end{itemize}
\end{frame}
\begin{frame}[fragile]{Citations and referencing with BibTeX}
  \begin{itemize}
    \item References are maintained in the BibTeX standard format:
  \end{itemize}
  \begin{figure}[h]
    \centering
    \begin{minted}[fontsize=\footnotesize]{bibtex}
% Save in a different file in the same directory
% as the document as mybib.bib
@Misc{RedHat2008,
author = {RedHat},
title  = {Fedora Project},
date   = {2008},
year   = {2008},
}

    \end{minted}
  \end{figure}
\end{frame}
\begin{frame}[fragile]{Citations and referencing with BibTeX II}
  \begin{itemize}
    \item Most standard publishers provide BibTeX citation information for their articles (so does \href{https://scholar.google.com/}{Google Scholar}).
    \item A plethora of Free/Open source bibliography managers are available, almost all of which, support BibTeX\@:
      \begin{itemize}
        \item \href{http://www.jabref.org/}{JabRef}.
        \item \href{http://wikindx.sourceforge.net/}{Wikindx} (Web based).
        \item \href{https://en.wikipedia.org/wiki/Zotero}{Zotero} (Local and Web interfaces).
      \end{itemize}
  \end{itemize}
\end{frame}
\begin{frame}[fragile]{Citations and referencing with BibTeX III}
  \begin{figure}[h]
    \centering
    \begin{minted}[fontsize=\footnotesize]{latex}
% Cite a reference in the text:
\section{A reference}
The Fedora project community\cite{RedHat2008} is committed
to promoting Free/Open source.

% A list of citations
\bibliographystyle{plain}
\bibliography{mybib}
    \end{minted}
  \end{figure}
  \begin{itemize}
    \item Save, re-run \texttt{pdflatex}, view.
  \end{itemize}
\end{frame}
\begin{frame}[fragile]{Citations and referencing with BibTeX IV}
  \begin{figure}[h]
    \centering
    \begin{minted}[fontsize=\footnotesize]{console}
$ pdflatex example-doc
LaTeX Warning: There were undefined references.
    \end{minted}
  \end{figure}
  \begin{figure}[h]
    \centering
    \includegraphics[width=\textwidth,keepaspectratio]{20180110-latex-101-bibtex.png}
  \end{figure}
\end{frame}
\begin{frame}[fragile]{Citations and referencing with BibTeX V}
  \begin{itemize}
    \item Multiple passes are needed to generate the document. Simply:
      \begin{itemize}
        \item the locations of the citations are stored in the first pass,
        \item the bibliography is processed next,
        \item the locations are completed with the required text.
      \end{itemize}
  \end{itemize}
  \begin{figure}[h]
    \centering
    \begin{minted}[fontsize=\footnotesize]{latex}
$ pdflatex example-doc && bibtex example-doc \
    && pdflatex example-doc && pdflatex example-doc
    \end{minted}
  \end{figure}
  \begin{itemize}
    \item Tip: look at \href{https://mg.readthedocs.io/latexmk.html}{Latexmk} (Yes, it's available in Fedora).
  \end{itemize}
\end{frame}
\section{Collaborative writing}
\begin{frame}[c]{Using Git/Github}
  \begin{itemize}
    \item Plaintext: multiple people can work on different parts of the text together.
    \item Use the power of \href{http://phdcomics.com/comics/archive.php?comicid=1531}{version control}!
    \item Can follow the standard pull request model used commonly in software development nowadays.
      \pause{}
    \item A few tips:
      \begin{itemize}
        \item Write each sentence on a new line: this helps git to merge easily, since git looks fir differences between lines.
        \item Break the main text into smaller files using \texttt{include} or \texttt{input} commands\footnotemark.
      \end{itemize}
  \end{itemize}
  \footnotetext[6]{\href{https://tex.stackexchange.com/questions/246/when-should-i-use-input-vs-include}{Include vs input}}
\end{frame}
\begin{frame}[c,fragile]{Using Git/Github II}
  \begin{figure}[h]
    \centering
    \begin{minted}[fontsize=\footnotesize]{latex}
\begin{document}
% Let LaTeX format the title
\maketitle
% The text of the document goes here.

% sections in different files
\section{Our first section}
\LaTeX{} is easy!

\subsection{Lists}
\begin{itemize}
  \item An itemised list!
\end{itemize}

\begin{enumerate}
  \item A numbered list
\end{enumerate}


\section{Some maths}
\begin{equation}
  h^2 = b^2 + p^2
\end{equation}
where \(h\), \(b\), and \(p\) are the lengths of the hypotenuse, the base, and the perpendicular of a right angled triangle respectively.



\section{A reference}
The Fedora project community\cite{RedHat2008} is committed to promoting Free/Open source.




\bibliographystyle{plain}
\bibliography{mybib}

\end{document}
    \end{minted}
  \end{figure}
\end{frame}

\begin{frame}[c]{\LaTeX{} 101}
  \begin{center}
    \href{https://fedoraproject.org/wiki/Classroom}{fedoraproject.org/wiki/Classroom}\vspace{0.2cm}

    \href{https://docs.fedoraproject.org/en-US/neurofedora/latex/}{docs.fedoraproject.org/en-US/neurofedora/latex/}\vspace{0.2cm}

    \href{http://creativecommons.org/licenses/by-sa/4.0/}{Creative Commons Attribution-ShareAlike 4.0 International License}.\vspace{0.2cm}

    \ccbysa{}
  \end{center}
\end{frame}
\end{document}
